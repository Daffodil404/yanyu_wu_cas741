% THIS DOCUMENT IS TAILORED TO REQUIREMENTS FOR SCIENTIFIC COMPUTING.  IT SHOULDN'T
% BE USED FOR NON-SCIENTIFIC COMPUTING PROJECTS
\documentclass[12pt]{article}

\usepackage{amsmath, mathtools}
\usepackage{amsfonts}
\usepackage{amssymb}
\usepackage{graphicx}
\usepackage{colortbl}
\usepackage{xr}
\usepackage{hyperref}
\usepackage{longtable}
\usepackage{xfrac}
\usepackage{tabularx}
\usepackage{float}
\usepackage{siunitx}
\usepackage{booktabs}
\usepackage{caption}
\usepackage{pdflscape}
\usepackage{afterpage}

\usepackage[round]{natbib}

%\usepackage{refcheck}

\hypersetup{
  bookmarks=true,         % show bookmarks bar?
  colorlinks=true,       % false: boxed links; true: colored links
  linkcolor=red,          % color of internal links (change box color with linkbordercolor)
  citecolor=green,        % color of links to bibliography
  filecolor=magenta,      % color of file links
  urlcolor=cyan           % color of external links
}

\input{../Comments.text}
\input{../Common.text}

% For easy change of table widths
\newcommand{\colZwidth}{1.0\textwidth}
\newcommand{\colAwidth}{0.13\textwidth}
\newcommand{\colBwidth}{0.82\textwidth}
\newcommand{\colCwidth}{0.1\textwidth}
\newcommand{\colDwidth}{0.05\textwidth}
\newcommand{\colEwidth}{0.8\textwidth}
\newcommand{\colFwidth}{0.17\textwidth}
\newcommand{\colGwidth}{0.5\textwidth}
\newcommand{\colHwidth}{0.28\textwidth}

% Used so that cross-references have a meaningful prefix
\newcounter{defnum} %Definition Number
\newcommand{\dthedefnum}{GD\thedefnum}
\newcommand{\dref}[1]{GD\ref{#1}}
\newcounter{datadefnum} %Datadefinition Number
\newcommand{\ddthedatadefnum}{DD\thedatadefnum}
\newcommand{\ddref}[1]{DD\ref{#1}}
\newcounter{theorynum} %Theory Number
\newcommand{\tthetheorynum}{TM\thetheorynum}
\newcommand{\tref}[1]{TM\ref{#1}}
\newcounter{tablenum} %Table Number
\newcommand{\tbthetablenum}{TB\thetablenum}
\newcommand{\tbref}[1]{TB\ref{#1}}
\newcounter{assumpnum} %Assumption Number
\newcommand{\atheassumpnum}{A\theassumpnum}
\newcommand{\aref}[1]{A\ref{#1}}
\newcounter{goalnum} %Goal Number
\newcommand{\gthegoalnum}{GS\thegoalnum}
\newcommand{\gsref}[1]{GS\ref{#1}}
\newcounter{instnum} %Instance Number
\newcommand{\itheinstnum}{IM\theinstnum}
\newcommand{\iref}[1]{IM\ref{#1}}
\newcounter{reqnum} %Requirement Number
\newcommand{\rthereqnum}{R\thereqnum}
\newcommand{\rref}[1]{R\ref{#1}}
\newcounter{nfrnum} %NFR Number
\newcommand{\rthenfrnum}{NFR\thenfrnum}
\newcommand{\nfrref}[1]{NFR\ref{#1}}
\newcounter{lcnum} %Likely change number
\newcommand{\lthelcnum}{LC\thelcnum}
\newcommand{\lcref}[1]{LC\ref{#1}}

\renewcommand{\progname}{Integrating NCRF into existing pipeline}

\usepackage{fullpage}

\newcommand{\deftheory}[9][Not Applicable]
{
  \newpage
  \noindent \rule{\textwidth}{0.5mm}

  \paragraph{RefName: } \textbf{#2} \phantomsection
  \label{#2}

  \paragraph{Label:} #3

  \noindent \rule{\textwidth}{0.5mm}

  \paragraph{Equation:}

  #4

  \paragraph{Description:}

  #5

  \paragraph{Notes:}

  #6

  \paragraph{Source:}

  #7

  \paragraph{Ref.\ By:}

  #8

  \paragraph{Preconditions for \hyperref[#2]{#2}:}
  \label{#2_precond}

  #9

  \paragraph{Derivation for \hyperref[#2]{#2}:}
  \label{#2_deriv}

  #1

  \noindent \rule{\textwidth}{0.5mm}

}

\begin{document}

\title{Software Requirements Specification for \progname}
\author{Yanyu Wu}
\date{\today}

\maketitle

~\newpage

\pagenumbering{roman}

\tableofcontents

~\newpage

\section*{Revision History}

\begin{tabularx}{\textwidth}{p{3cm}p{2cm}X}
  \toprule {\bf Date} & {\bf Version} & {\bf Notes}\\
  \midrule
  Feb 2,2026 & 1.1 & Initial Draft\\
  \bottomrule
\end{tabularx}

~\\

~\newpage

\section{Reference Material}

This section records information for easy reference.

\subsection{Table of Units}

Throughout this document SI (Syst\`{e}me International d'Unit\'{e}s) is employed
as the unit system.  In addition to the basic units, several derived units are
used as described below.  For each unit, the symbol is given followed by a
description of the unit and the SI name.
~\newline

\renewcommand{\arraystretch}{1.2}
%\begin{table}[ht]
\noindent
\begin{tabular}{l l l}
  \toprule
  \textbf{symbol} & \textbf{unit} & \textbf{SI}\\
  \midrule
  \si db & sound & decibel\\
  \si{\second} & time & second\\
  \si ms & time & millisecond\\
  \si Hz & frequency & hertz \\
  \bottomrule
\end{tabular}
%  \caption{Provide a caption}
%\end{table}

% \plt{Only include the units that your SRS actually uses.}

% \plt{Derived units, like newtons, pascal, etc, should show their derivation
%     (the units they are derived from) if their constituent units are in the
%     table of units (that is, if the units they are derived from are used in the
%     document).  For instance, the derivation of pascals as
%     $\si{\pascal}=\si{\newton\per\square\meter}$ is shown if newtons and m are
%     both in the table.  The derivations of newtons would not be shown if kg and
%     s are not both in the table.}

% \plt{The symbol for units named after people use capital letters, but the name
%   of the unit itself uses lower case.  For instance, pascals use the symbol Pa,
%   watts use the symbol W, teslas use the symbol T, newtons use the symbol N,
%   etc.  The one exception to this is degree Celsius.  Details on writing metric
%   units can be found on the
%   \href{https://www.nist.gov/pml/weights-and-measures/writing-metric-units}
%   {NIST} web-page.}

\subsection{Table of Symbols}

The table that follows summarizes the symbols used in this document along with
their units.  The choice of symbols was made to be consistent with the heat
transfer literature and with existing documentation for solar water heating
systems.  The symbols are listed in alphabetical order.

\renewcommand{\arraystretch}{1.2}
%\noindent \begin{tabularx}{1.0\textwidth}{l l X}
\noindent
\begin{longtable*}{l l p{12cm}} \toprule
  \textbf{symbol} & \textbf{unit} & \textbf{description}\\
  \midrule
  t & s/ms & basic time unit \\
  $l$ & s/ms & time lag, relative delay between stimulus and neural response\\
  m & - & cortical source index \\
  $e_t$ & db & audio stimulus at time t \\
  $r_m$ & m &  the right-anterior-superior (RAS) coordinates of the $m^{th}$ dipole \\
  $j_{m,t}$ & A·m & Dipole current vector at source m and time t\\
  $v_{m,t}$ & A·m & stimulus-independent background activity \\
  $\tau_{m,i,l}$ & - & 3D vector-valued TRF at source m,defined as NCRF.\\
  \bottomrule
\end{longtable*}
% \plt{Use your problems actual symbols.  The si package is a good idea to use for
%   units.}

\subsection{Abbreviations and Acronyms}

\renewcommand{\arraystretch}{1.2}
\begin{tabular}{l l}
  \toprule
  \textbf{symbol} & \textbf{description}\\
  \midrule
  A & Assumption\\
  DD & Data Definition\\
  GD & General Definition\\
  GS & Goal Statement\\
  IM & Instance Model\\
  LC & Likely Change\\
  PS & Physical System Description\\
  R & Requirement\\
  SRS & Software Requirements Specification\\
  \progname{} & analyzing pipeline for eeg data via NCRF method\\
  TM & Theoretical Model\\
  \bottomrule
\end{tabular}\\

% \plt{Add any other abbreviations or acronyms that you add}

% \subsection{Mathematical Notation}

% \plt{This section is optional, but should be included for projects that make use
%   of notation to convey mathematical information.  For instance, if typographic
%   conventions (like bold face font) are used to distinguish matrices, this
%   should be stated here.  If symbols are used to show mathematical operations,
%   these should be summarized here.  In some cases the easiest way to summarize
%   the notation is to point to a text or other source that explains the
%   notation.}

% \plt{This section was added to the template because some students use very
%   domain specific notation.  This notation will not be readily understandable to
%   people outside of your domain.  It should be explained.}

\newpage

% \pagenumbering{arabic}

% \plt{This SRS template is based on \citet{SmithAndLai2005, SmithEtAl2007,
%   SmithAndKoothoor2016}.  It will get you started.  You should not modify the
%   section headings, without first discussing the change with the course
%   instructor.  Modification means you are not following the template, which
%   loses some of the advantage of a template, especially standardization.
%   Although the bits shown below do not include type information, you may need to
%   add this information for your problem.  If you are unsure, please can ask the
%   instructor.}

% \plt{Feel free to change the appearance of the report by modifying the LaTeX
%   commands.}

% \plt{This template document assumes that a single program is being documented.
%   If you are documenting a family of models, you should start with a commonality
%   analysis.  A separate template is provided for this.  For program
%   families you should look at \cite{Smith2006, SmithMcCutchanAndCarette2017}.
%   Single family member programs are often programs based on a single physical
%   model.  General purpose tools are usually documented as a family.  Families of
%   physical models also come up.}

% \plt{The SRS is not generally written, or read, sequentially.  The SRS is a
%   reference document.  It is generally read in an ad hoc order, as the need
%   arises.  For writing an SRS, and for reading one for the first time, the
%   suggested order of sections is:
% \begin{itemize}
% \item Goal Statement
% \item Instance Models
% \item Requirements
% \item Introduction
% \item Specific System Description
% \end{itemize}
% }

% \plt{Guiding principles for the SRS document:
% \begin{itemize}
% \item Do not repeat the same information at the same abstraction level.  If
%   information is repeated, the repetition should be at a different abstraction
%   level.  For instance, there will be overlap between the scope section and the
%   assumptions, but the scope section will not go into as much detail as the
%   assumptions section.
% \end{itemize}
% }

% \plt{The template description comments should be disabled before submitting this
%   document for grading.}

% \plt{You can borrow any wording from the text given in the template.  It is part
%   of the template, and not considered an instance of academic integrity.  Of
%   course, you need to cite the source of the template.}

% \plt{When the documentation is done, it should be possible to trace back to the
%   source of every piece of information.  Some information will come from
%   external sources, like terminology.  Other information will be derived, like
%   General Definitions.}

% \plt{An SRS document should have the following qualities: unambiguous,
%   consistent, complete, validatable, abstract and traceable.}

% \plt{The overall goal of the SRS is that someone that meets the Characteristics
%   of the Intended Reader (Section~\ref{sec_IntendedReader}) can learn,
%   understand and verify the captured domain knowledge.  They should not have to
%   trust the authors of the SRS on any statements.  They should be able to
%   independently verify/derive every statement made.}

\section{Introduction}

The ability to understand how the human brain processes continuous auditory stimuli, such as speech, is a central problem in cognitive neuroscience. Non-invasive measurements, such as magnetoencephalography (MEG) and electroencephalogram (EEG), provide high temporal resolution recordings of neural activities.
Typically, modeling how the brain's cortical sources respond to time-varying stimuli from these measurement statistics is challenging.
However, the Neuro-Current Response Function (NCRF) directly models the 3D cortical dipole currents as linear filters of stimulus features, avoiding the need for separate source reconstruction or fixed dipole orientations. By combining vector-valued impulse responses with the forward model and regularized estimation, NCRF enables accurate prediction of neural responses while flexibly adapting to each source’s spatial characteristics.

\subsection{Purpose of Document}
This section provides an overview of the Software Requirements Specification (SRS) for the NCRF pipeline. It explains the purpose of this document, the scope of requirements, the intended reader characteristics, and the organization of the document.

% \plt{This section summarizes the purpose of the SRS document.  It does not focus
%   on the problem itself.  The problem is described in the ``Problem
%   Description'' section (Section~\ref{Sec_pd}).  The purpose is for the document
%   in the context of the project itself, not in the context of this course.
%   Although the ``purpose'' of the document is to get a grade, you should not
%   mention this.  Instead, ``fake it'' as if this is a real project.  The purpose
%   section will be similar between projects.  The purpose of the document is the
%   purpose of the SRS, including communication, planning for the design stage,
%   etc.}

\subsection{Scope of Requirements}
This project focuses on integrating the Neuro-Current Response Function (NCRF) framework into an existing Temporal Response Function (TRF) experimental pipeline.
The following items are included in the scope:
\begin{itemize}
  \item Integration of NCRF concepts into the TRF pipeline to enable source-level response estimation.
  \item Adaptation of input and intermediate data formats to ensure compatibility with the NCRF-based modeling approach.
\end{itemize}
The following aspects are explicitly out of scope:
\begin{itemize}
  \item Collection, acquisition, or preprocessing of MEG or EEG data.
  \item Redesign or retraining of the existing TRF pipeline core implementation.
\end{itemize}

% \plt{Modelling the real world requires simplification.  The full complexity of
%   the actual physics, chemistry, biology is too much for existing models, and
%   for existing computational solution techniques.  Rather than say what is in
%   the scope, it is usually easier to say what is not.  You can think of it as
%   the scope is initially everything, and then it is constrained to create the
%   actual scope.  For instance, the problem can be restricted to 2 dimensions, or
%   it can ignore the effect of temperature (or pressure) on the material
%   properties, etc.}

% \plt{The scope section is related to the assumptions section
%   (Section~\ref{sec_assumpt}).  However, the scope and the assumptions are not
%   at the same level of abstraction.  The scope is at a high level.  The focus is
%   on the ``big picture'' assumptions.  The assumptions section lists, and
%   describes, all of the assumptions.}

% \plt{The scope section is relevant for later determining typical values of inputs. The scope should make it clear what inputs are reasonable to expect. This is a distinction between scope and context (context is a later section).  Scope affects the inputs while context affects how the software will be used.}

\subsection{Characteristics of Intended Reader} \label{sec_IntendedReader}

The intended readers of this Software Requirements Specification should have an undergraduate-level understanding of programming and math.
The users of the NCRF pipeline may have a lower level of expertise, as explained in Section~\ref{SecUserCharacteristics}.

% \plt{This section summarizes the skills and knowledge of the readers of the
%   SRS.  It does NOT have the same purpose as the ``User Characteristics''
%   section (Section~\ref{SecUserCharacteristics}).  The intended readers are the
%   people that will read, review and maintain the SRS.  They are the people that
%   will conceivably design the software that is intended to meet the
%   requirements.  The user, on the other hand, is the person that uses the
%   software that is built.  They may never read this SRS document.  Of course,
%   the same person could be a ``user'' and an ``intended reader.''}

% \plt{The intended reader characteristics should be written as unambiguously and
%   as specifically as possible.  Rather than say, the user should have an
%   understanding of physics, say what kind of physics and at what level.  For
%   instance, is high school physics adequate, or should the reader have had a
%   graduate course on advanced quantum mechanics?}

\subsection{Organization of Document}
The organization of the whole SRS document follows the template for scientific computing software proposed by 
{\cite{SmithAndLai2005, SmithEtAl2007,SmithAndKoothoor2016}}.\\

The document is intended to be read selectively rather than sequentially. \\

For readers who wish to obtain a high-level understanding of the problem and the
intended capabilities of the system, the recommended reading order is as follows:

\begin{itemize}

  \item \textbf{\hyperref[Sec_pd]{Problem Description}} , which presents the scientific
  problem addressed by the system and the context of NCRF-based analysis.
  \item \textbf{\hyperref[sec_goalstatements]{Goal Statements}}, which summarize the objectives
  the system is required to achieve.
  \item \textbf{\hyperref[sec_scope]{Scope of Requirements}} and
  \textbf{\hyperref[sec_IntendedReader]{Characteristics of Intended Reader}} which clarify the
  applicability and assumed background knowledge.
  \item \textbf{\hyperref[sec_gensysdesc]{General System Description}}, which outlines the system
  boundaries, user interactions, and external dependencies.
  \item \textbf{\hyperref[sec_assumpt]{Assumptions}}, which describe the simplifying
  conditions under which the scientific models are valid.
\end{itemize}

Readers interested in the mathematical formulation and verification of the system
should then consult the \textbf{\hyperref[sub_sec_scs]{Solution Characteristics Specification}}
, with particular attention to the \textbf{\hyperref[sec_theoretical]{Theoretical Models}}
and \textbf{\hyperref[sec_instance]{Instance Models}}. \\

The remaining sections, including requirements, traceability matrices, and auxiliary
constants, are primarily referential and are intended to support validation,
maintenance, and future modification of the system.

\newpage
% \plt{This section provides a roadmap of the SRS document.  It will help the
%   reader orient themselves.  It will provide direction that will help them
%   select which sections they want to read, and in what order.  This section will
%   be similar between project.}

% \plt{Include a reference to the template (\citet{SmithAndLai2005, SmithEtAl2007,
%   SmithAndKoothoor2016}) you are using in the documentation in the Organization of
%   the Document section.}


\section{General System Description}

This section provides general information about the system.  It identifies the
interfaces between the system and its environment, describes the user
characteristics and lists the system constraints.
% \plt{This text can likely be borrowed verbatim.}

% \plt{The purpose of this section is to provide general information about the
%   system so the specific requirements in the next section will be easier to
%   understand. The general system description section is designed to be
%   changeable independent of changes to the functional requirements documented in
%   the specific system description. The general system description provides a
%   context for a family of related models.  The general description can stay the
%   same, while specific details are changed between family members.}

\subsection{System Context}

The Figure1 below indicates the system context about this project.
The user could be a researcher or graduate student works on neuro-science, who will provide time-series stimuli and related neuro data and it has been represented by the circle in Fig 1. The rectangle represents the software system (NCRF pipeline) and the the arrow indicates the workflow of data.

% \plt{Your system context will include a figure that shows the abstract view of
%   the software.  Often in a scientific context, the program can be viewed
%   abstractly following the design pattern of Inputs $\rightarrow$ Calculations
%   $\rightarrow$ Outputs.  The system context will therefore often follow this
%   pattern.  The user provides inputs, the system does the calculations, and then
%   provides the outputs to the user.  The figure should not show all of the
%   inputs, just an abstract view of the main categories of inputs (like material
%   properties, geometry, etc.).  Likewise, the outputs should be presented from
%   an abstract point of view.  In some cases the diagram will show other external
%   entities, besides the user.  For instance, when the software product is a
%   library, the user will be another software program, not an actual end user.
%   If there are system constraints that the software must work with external
%   libraries, these libraries can also be shown on the System Context diagram.
%   They should only be named with a specific library name if this is required by
%   the system constraint.}

\begin{figure}[h!]
  \begin{center}
    \includegraphics[width=0.6\textwidth]{SystemContextFigure}
    \caption{System Context}
    \label{Fig_SystemContext}
  \end{center}
\end{figure}

% \plt{For each of the entities in the system context diagram its responsibilities
%   should be listed.  Whenever possible the system should check for data quality,
%   but for some cases the user will need to assume that responsibility.  The list
%   of responsibilities should be about the inputs and outputs only, and they
%   should be abstract.  Details should not be presented here.  However, the
%   information should not be so abstract as to just say ``inputs'' and
%   ``outputs''.  A summarizing phrase can be used to characterize the inputs.
%   For instance, saying ``material properties'' provides some information, but it
%   stays away from the detail of listing every required properties.}

\begin{itemize}
  \item User Responsibilities:
    \begin{itemize}
      \item Provide legal input data, ensuring the data type is supported by NCRF pipeline.
      \item Use the existing methods of NCRF pipeline, obeying the usage rules.
    \end{itemize}
    % \item \progname{} Responsibilities:
  \item NCRF pipeline Responsibilities:
    \begin{itemize}
      \item \textbf{Be Correct}: Generate outputs that accurately reflect the processing of inputs according to requirements.
      \item \textbf{Be Reliable}: Ensure consistent operation and reproducibility over time.
      \item \textbf{Be Robust:} Detect and handle errors or exceptional cases gracefully.
      \item \textbf{Keep Good Performance}: Process inputs efficiently to produce timely outputs.
      \item \textbf{Guarantee Data Quality}: Validate input data where possible and provide informative feedback to the user.
    \end{itemize}
\end{itemize}

The software is intended to be used in neuroscience research, where researchers analyze preprocessed brain imaging data and corresponding time-series stimuli. It will be used for a mission-critical applications.

% \plt{Identify in what context the software will typically be used.  Is it for
% exploration? education? engineering work? scientific work?. Identify whether it
% will be used for mission-critical or safety-critical applications.} \plt{This
% additional context information is needed to determine how much effort should be
% devoted to the rationale section.  If the application is safety-critical, the
% bar is higher.  This is currently less structured, but analogous to, the idea to
% the Automotive Safety Integrity Levels (ASILs) that McSCert uses in their
% automotive hazard analyses.}

\subsection{User Characteristics} \label{SecUserCharacteristics}
The users of NCRF pipeline should have a basic understanding of Neuro Science, such as basic neuro data structure, basic neuro science analysis algorithm. Programming ability is also required for end users, to accessing the pipeline by python.

% \plt{This section summarizes the knowledge/skills expected of the user.
%   Measuring usability, which is often a required non-function requirement,
%   requires knowledge of a typical user.  As mentioned above, the user is a
%   different role from the ``intended reader,'' as given in
%   Section~\ref{sec_IntendedReader}.  As in Section~\ref{sec_IntendedReader}, the
%   user characteristics should be specific an unambiguous.  For instance, ``The
%   end user of \progname{} should have an understanding of undergraduate Level 1
%   Calculus and Physics.''}

\subsection{System Constraints}
The system should be built based on the existing TRF Experiment pipeline and the related NCRF modules of Eelbrain. The pipeline will be built via Python and interactive functions should be in Jupyter Notebook.
% \plt{System constraints differ from other type of requirements because they
%   limit the developers' options in the system design and they identify how the
%   eventual system must fit into the world. This is the only place in the SRS
%   where design decisions can be specified.  That is, the quality requirement for
%   abstraction is relaxed here.  However, system constraints should only be
%   included if they are truly required.}

\newpage

\section{Specific System Description}

This section first presents the problem description, which gives a high-level
view of the problem to be solved.  This is followed by the solution characteristics
specification, which presents the assumptions, theories, definitions and finally
the instance models.
% \plt{Add any project specific details that are relevant
%   for the section overview.}

\subsection{Problem Description} \label{Sec_pd}

\progname{} is intended to solve the error propagation issue of traditional TRF-based pipeline. Using NCRF could allow spatial and temporal response properties to be estimated jointly, instead of sequentially.

%   ... \plt{What problem does your program solve?
% The description here should be in the problem space, not the solution space.}

\subsubsection{Terminology and  Definitions}

% \plt{This section is expressed in words, not with equations.  It provide the
%   meaning of the different words and phrases used in the domain of the problem.
% The terminology is used to introduce concepts from the world outside of the
% mathematical model  The terminology provides a real world connection to give the
% mathematical model meaning.}

This subsection provides a list of terms that are used in the subsequent sections and their meaning, with the purpose of reducing ambiguity and making it easier to correctly understand the requirements:

\begin{itemize}

  \item \textbf{TRF}: TRF is the short for (Temporal Response Function). It is an encoding model, that describes
    how the continous stimuli will be mapped to neural signal by brain.
  \item \textbf{NCRF}: NCRF is short for Neuro-Current Response Function. It is a source-level linear response function that models how cortical dipole currents in the brain respond over time to external stimulus features.
  \item \textbf{Dipole}: Dipole is a simplified model of a small localized neural current in the brain, representing  direction of electrical activity at a cortical source.

\end{itemize}

\subsubsection{Physical System Description} \label{sec_phySystDescrip}

% \plt{The purpose of this section is to clearly and unambiguously state the physical system that is to be modelled. Effective problem solving requires a
%   logical and organized approach. The statements on the physical system to be studied should cover enough information to solve the problem. The physical description involves element identification, where elements are defined as independent and separable items of the physical system. Some example elements include acceleration due to gravity, the mass of an object, and the size and
%   shape of an object. Each element should be identified and labelled, with their interesting properties specified clearly. The physical description can also include interactions of the elements, such as the following: i) the interactions between the elements and their physical environment; ii) the
%   interactions between elements; and, iii) the initial or boundary conditions.}

% \plt{The elements of the physical system do not have to correspond to an actual
% physical entity.  They can be conceptual.  This is particularly important when
% the documentation is for a numerical method. }

The physical system of \progname{}, as shown in Figure~?,
includes the following elements:

\begin{itemize}

  \item[PS1:] Fig 2 represents how the continous stimuli make brain response and how the response be captured.

    \begin{figure}[h!]
      \begin{center}
        \includegraphics[width=0.6\textwidth]{Stimulus2Signal}
        \caption{Physical system of continuous stimulus processing measured by EEG/MEG}
        \label{Fig_PS1}
      \end{center}
    \end{figure}

  \item[PS2:] Fig 3 represents the workflow of NCRF pipeline. It receives stimuli and produces the spatial-temporal matched brain current response.

    \begin{figure}[h!]
      \begin{center}
        \includegraphics[width=0.6\textwidth]{NCRF}
        \caption{Conceptual model of stimulus-to-cortex mapping using NCRF}
        \label{Fig_PS2}
      \end{center}
    \end{figure}

\end{itemize}

% \plt{A figure here makes sense for most SRS documents}

% \begin{figure}[h!]
% \begin{center}
% %\rotatebox{-90}
% {
%  \includegraphics[width=0.5\textwidth]{<FigureName>}
% }
% \caption{\label{<Label>} <Caption>}
% \end{center}
% \end{figure}

\subsubsection{Goal Statements}

% \plt{The goal statements refine the ``Problem Description''
%   (Section~\ref{Sec_pd}).  A goal is a functional objective the system under
%   consideration should achieve. Goals provide criteria for sufficient
%   completeness of a requirements specification and for requirements
%   pertinence. Goals will be refined in Section “Instanced Models”
%   (Section~\ref{sec_instance}). Large and complex goals should be decomposed
%   into smaller sub-goals.  The goals are written abstractly, with a minimal
%   amount of technical language.  They should be understandable by non-domain
%   experts.}

\noindent Given the time-series stimuli, the goal statements are:

\begin{itemize}

    % \item[GS\refstepcounter{goalnum}\thegoalnum \label{G_meaningfulLabel}:] \plt{One
    %     sentence description of the goal.  There may be more than one.  Each Goal
    %     should have a meaningful label.}

  \item[GS1:] Accept continuous stimulus signals and corresponding EEG/MEG recordings.
  \item[GS2:] Estimate source-level neuro-current responses using NCRF methodology.
  \item[GS3:] Format results to be compatible with existing pipeline outputs and analysis workflows.

\end{itemize}

\subsection{Solution Characteristics Specification}

% \plt{This section specifies the information in the solution domain of the system
%   to be developed. This section is intended to express what is required in
%   such a way that analysts and stakeholders get a clear picture, and the
%   latter will accept it. The purpose of this section is to reduce the problem
%   into one expressed in mathematical terms. Mathematical expertise is used to
%   extract the essentials from the underlying physical description of the
%   problem, and to collect and substantiate all physical data pertinent to the
%   problem.}

% \plt{This section presents the solution characteristics by successively refining
%   models.  It starts with the abstract/general Theoretical Models (TMs) and
%   refines them to the concrete/specific Instance Models (IMs).  If necessary
%   there are intermediate refinements to General Definitions (GDs).  All of these
%   refinements can potentially use Assumptions (A) and Data Definitions (DD).
%   TMs are refined to create new models, that are called GMs or IMs. DDs are not
%   refined; they are just used. GDs and IMs are derived, or refined, from other
%   models. DDs are not derived; they are just given. TMs are also just given, but
%   they are refined, not used.  If a potential DD includes a derivation, then
%   that means it is refining other models, which would make it a GD or an IM.}

% \plt{The above makes a distinction between ``refined'' and ``used.'' A model is
%   refined to another model if it is changed by the refinement. When we change a
%   general 3D equation to a 2D equation, we are making a refinement, by applying
%   the assumption that the third dimension does not matter. If we use a
%   definition, like the definition of density, we aren't refining, or changing
%   that definition, we are just using it.}

% \plt{The same information can be a TM in one problem and a DD in another.  It is
%   about how the information is used.  In one problem the definition of
%   acceleration can be a TM, in another it would be a DD.}

% \plt{There is repetition between the information given in the different chunks
%   (TM, GDs etc) with other information in the document.  For instance, the
%   meaning of the symbols, the units etc are repeated.  This is so that the
%   chunks can stand on their own when being read by a reviewer/user.  It also
%   facilitates reuse of the models in a different context.}

% \noindent \plt{The relationships between the parts of the document are show in
%   the following figure.  In this diagram ``may ref'' has the same role as
%   ``uses'' above.  The figure adds ``Likely Changes,'' which are able to
%   reference (use) Assumptions.}

% \begin{figure}[H]
%   \includegraphics[scale=0.9]{RelationsBetweenTM_GD_IM_DD_A.pdf}
% \end{figure}

% The instance models that govern \progname{} are presented in
% Subsection~\ref{sec_instance}.  The information to understand the meaning of the
% instance models and their derivation is also presented, so that the instance models can be verified.

% \subsubsection{Types}

% \plt{This section is optional. Defining types can make the document easier to
% understand.}

% \subsubsection{Scope Decisions}

% \plt{This section is optional.}
% \subsubsection{Modelling Decisions}

% \plt{This section is optional.}

\subsubsection{Assumptions} \label{sec_assumpt}

% \plt{The assumptions are a refinement of the scope.  The scope is general, where
%   the assumptions are specific.  All assumptions should be listed, even those
%   that domain experts know so well that they are rarely (if ever) written down.}
% \plt{The document should not take for granted that the reader knows which
%   assumptions have been made. In the case of unusual assumptions, it is
%   recommended that the documentation either include, or point to, an explanation
%   and justification for the assumption.}
% \plt{If it helps with the organization and understandability, the assumptions
% can be presented as sub sections.  The following sub-sections are options:
% background theory assumptions, helper theory assumptions, generic theory
% assumptions, problem specific assumptions, and rationale assumptions}

This section simplifies the original problem and helps in developing the
theoretical model by filling in the missing information for the physical system.
The numbers given in the square brackets refer to the theoretical model [TM],
general definition [GD], data definition [DD], instance model [IM], or likely
change [LC], in which the respective assumption is used.

\begin{itemize}

    % \item[A\refstepcounter{assumpnum}\theassumpnum \label{A_meaningfulLabel}:]
    %   \plt{Short description of each assumption.  Each assumption
    %     should have a meaningful label.  Use cross-references to identify the
    %     appropriate traceability to TM, GD, DD etc., using commands like dref, ddref
    %     etc.  Each assumption should be atomic - that is, there should not be an
    %     explicit (or implicit) ``and'' in the text of an assumption.}

  \item[A\refstepcounter{assumpnum}\theassumpnum \label{A1}:]
  Preprocessed, time-aligned stimulus features and EEG/MEG recordings are available as inputs.

  \item[A\refstepcounter{assumpnum}\theassumpnum \label{A2}:]
  Cortical source responses are modeled using linear temporal filters; nonlinear dynamics are not considered.

  \item[A\refstepcounter{assumpnum}\theassumpnum \label{A3}:]
  The cortical source space and the corresponding forward model are assumed to be fixed and known.

  \item[A\refstepcounter{assumpnum}\theassumpnum \label{A4}:]
  The brain response to a stimulus is not instantaneous; it occurs with a non-zero time lag.
\end{itemize}

\subsubsection{Theoretical Models}\label{sec_theoretical}

% \plt{Theoretical models are sets of abstract mathematical equations or axioms
%   for solving the problem described in Section ``Physical System Description''
%   (Section~\ref{sec_phySystDescrip}). Examples of theoretical models are
%   physical laws, constitutive equations, relevant conversion factors, etc.}

% \plt{Optionally the theory section could be divided into subsections to provide
% more structure and improve understandability and reusability.  Potential
% subsections include the following: Context theories, background theories, helper
% theories, generic theories, problem specific theories, final theories and
% rationale theories.}

This section focuses on the general equations and laws that NCRF pipeline is based
on. Under the linear temporal response assumption (A3), the stimulus-driven neural response is modeled as a temporal convolution.
% \plt{Modify the examples below for your problem, and add additional models as appropriate.}

~\newline

\noindent
\deftheory
% #2 refname of theory
{TM_NCRF_Model}
% #3 label
{Stimulus-to-Source Model (NCRF Model)}
% #4 equation
{\ensuremath{\mathbf{j}_m(t)=\sum_{\ell=0}^{L-1}\boldsymbol{\tau}_m(\ell)\, e(t-\ell)+\mathbf{v}_m(t)}}
% #5 description
{%
  The NCRF model defines how the stimulus feature drives the neural source current at cortical source location $m$.
  Here, $\mathbf{j}_m(t)$ is the (vector-valued) equivalent current dipole moment at source $m$ (SI unit: \si{\ampere\metre}).
  The term $e(t-\ell)$ is the stimulus feature at lag $\ell$ (feature-dependent unit; commonly dimensionless or arbitrary units).
  The vector-valued kernel $\boldsymbol{\tau}_m(\ell)$ is the neuro-current response function (NCRF) at source $m$ and lag $\ell$
  (unit: \si{\ampere\metre} per stimulus unit), and $L$ is the number of discrete lags.
  $\mathbf{v}_m(t)$ represents stimulus-independent background source activity (same unit as $\mathbf{j}_m(t)$).
}
% #6 Notes
{%
  None.
}
% #7 Source
{\citep{DasBrodbeckSimonBabadi2020}}
% #8 Referenced by
{%
  None.
}
% #9 Preconditions
{%
  None.
}

~\newline

\subsubsection{General Definitions}\label{sec_gendef}

% \plt{General Definitions (GDs) are a refinement of one or more TMs, and/or of
%   other GDs.  The GDs are less abstract than the TMs.  Generally the reduction
%   in abstraction is possible through invoking (using/referencing) Assumptions.
%   For instance, the TM could be Newton's Law of Cooling stated abstracting.  The
%   GD could take the general law and apply it to get a 1D equation.}

This section collects the laws and equations that will be used in building the
instance models.

% \plt{Some projects may not have any content for this section, but the section
%   heading should be kept.}  \plt{Modify the examples below for your problem, and
%   add additional definitions as appropriate.}

~\newline

\noindent
\begin{minipage}{\textwidth}
  \renewcommand*{\arraystretch}{1.5}

  \begin{tabular}{| p{\colAwidth} | p{\colBwidth}|}
    \hline
    \rowcolor[gray]{0.9}
    Number& GD\refstepcounter{defnum}\thedefnum \label{GD_StimulusDriven}\\
    \hline
    Label &\bf Stimulus-Driven Dipole Response \\
    \hline
    % Units&$MLt^{-3}T^0$\\
    % \hline
    SI Units & \si{\ampere\metre} \\
    \hline
    Equation & \[
      f_{m,i}(t)
      =
      \sum_{\ell=0}^{L-1}
      \tau_{m,i,\ell}\, e(t-\ell)
    \]
    \\
    \hline
    Description &
    $f_{m,i}(t)$ denotes the stimulus-driven component of the cortical dipole current
    at source location $m$, orientation $i$, and time $t$. \\
    & It is defined as a finite-lag linear convolution between the stimulus feature
    signal $e(t)$ and the neuro-current response function (NCRF) coefficients
    $\tau_{m,i,\ell}$. \\
    & This definition is obtained by isolating the stimulus-dependent term in the
    NCRF theoretical model and assumes a discrete-time representation of the stimulus.
    \\
    \hline
    Source & M:NCRF Model \cite{DasBrodbeckSimonBabadi2020}\\
    \hline
    Ref.\ By & \ddref{FluxCoil}, \ddref{FluxPCM}\\
    \hline
  \end{tabular}

  \begin{tabular}{| p{\colAwidth} | p{\colBwidth}|}
    \hline
    \rowcolor[gray]{0.9}
    Number& GD\refstepcounter{defnum}\thedefnum \label{GD_BackgroundActivity}\\
    \hline
    Label &\bf Background Dipole Activity \\
    \hline
    % Units&$MLt^{-3}T^0$\\
    % \hline
    SI Units & \si{\ampere\metre} \\
    \hline
    Equation & $v_{m,t}=v_{m}(t)$ \\
    \hline
    Description &
    $v_{m,t}$ represents stimulus-independent background neural activity of source $m$ at time $t$. This term accounts for intrinsic neural dynamics not explained by the stimulus-driven component.

    It represents intrinsic neural dynamics and other unmodeled processes that are not captured by the stimulus-driven response.
    \\
    \hline
    Source & Citation here \\
    \hline
    Ref.\ By & \ddref{FluxCoil}, \ddref{FluxPCM}\\
    \hline
  \end{tabular}
\end{minipage}\\

% \subsubsection*{Detailed derivation of simplified rate of change of temperature}

% \plt{This may be necessary when the necessary information does not fit in the
%   description field.}
% \plt{Derivations are important for justifying a given GD.  You want it to be
%   clear where the equation came from.}

\subsubsection{Data Definitions}\label{sec_datadef}

% \plt{The Data Definitions are definitions of symbols and equations that are
%   given for the problem.  They are not derived; they are simply used by other
%   models.  For instance, if a problem depends on density, there may be a data
%   definition for the equation defining density.  The DDs are given information
%   that you can use in your other modules.}

% \plt{All Data Definitions should be used (referenced) by at least one other
%   model.}

This section collects and defines all the data needed to build the instance
models. The dimension of each quantity is also given.
%  \plt{Modify the examples
%   below for your problem, and add additional definitions as appropriate.}

~\newline

\noindent
\begin{minipage}{\textwidth}
  \renewcommand*{\arraystretch}{1.5}
  \begin{tabular}{| p{\colAwidth} | p{\colBwidth}|}
    \hline
    \rowcolor[gray]{0.9}
    Number& DD\refstepcounter{datadefnum}\thedatadefnum \label{FluxCoil}\label{DD_NCRF_Coefficients}\\
    \hline
    Label& \bf Neuro-Current Response Function (NCRF) Coefficients \\
    \hline
    Symbol &$\tau_{m,i}$\\
    \hline
    % Units& $Mt^{-3}$\\
    % \hline
    SI Units & N/A \\
    \hline
    Equation&$\tau_{m,i}:=[\tau_{m,i,0},\tau_{m,i,1},\cdotp\cdotp\cdotp,\tau_{m,i,L-1}]^\top$\\
    \hline
    Description &
    The neuro-current response function (NCRF) coefficients $\tau_{m,i}$ are a finite set of parameters associated with cortical source m and orientation i.

    They define the temporal weighting applied to the stimulus feature over a fixed lag window of length L.

    These coefficients are treated as given data by the instance models and are used to construct the stimulus-driven component of the neural source current.
    \\
    \hline
    Sources& Citation here \\
    \hline
    Ref.\ By & \iref{ewat}\\
    \hline
  \end{tabular}
\end{minipage}\\

% \subsubsection{Data Types}\label{sec_datatypes}

% \plt{This section is optional.  In many scientific computing programs it isn't
%   necessary, since the inputs and outpus are straightforward types, like reals,
%   integers, and sequences of reals and integers.  However, for some problems it
%   is very helpful to capture the type information.}

% \plt{The data types are not derived; they are simply stated and used by other
%   models.}

% \plt{All data types must be used by at least one of the models.}

% \plt{For the mathematical notation for expressing types, the recommendation is
%   to use the notation of~\citet{HoffmanAndStrooper1995}.}

% This section collects and defines all the data types needed to document the
% models. \plt{Modify the examples below for your problem, and add additional
%   definitions as appropriate.}

% ~\newline

% \noindent
% \begin{minipage}{\textwidth}
% \renewcommand*{\arraystretch}{1.5}
% \begin{tabular}{| p{\colAwidth} | p{\colBwidth}|}
%   \hline
%   \rowcolor[gray]{0.9}
%   Type Name & Name for Type\\
%   \hline
%   Type Def & mathematical definition of the type\\
%   \hline
%   Description & description here
%   \\
%   \hline
%   Sources & Citation here, if the type is borrowed from another source\\
%   \hline
% \end{tabular}
% \end{minipage}\\

\subsubsection{Instance Models} \label{sec_instance}

% \plt{The motivation for this section is to reduce the problem defined in
%   ``Physical System Description'' (Section~\ref{sec_phySystDescrip}) to one
%   expressed in mathematical terms. The IMs are built by refining the TMs and/or
%   GDs.  This section should remain abstract.  The SRS should specify the
%   requirements without considering the implementation.}

This section transforms the problem defined in Section~\ref{Sec_pd} into
one which is expressed in mathematical terms. It uses concrete symbols defined
in Section~\ref{sec_datadef} to replace the abstract symbols in the models
identified in Sections~\ref{sec_theoretical} and~\ref{sec_gendef}.

% The goals \plt{reference your goals} are solved by \plt{reference your instance
%   models}.  \plt{other details, with cross-references where appropriate.}
% \plt{Modify the examples below for your problem, and add additional models as
%   appropriate.}

~\newline

%Instance Model 1

\noindent
\begin{minipage}{\textwidth}
  \renewcommand*{\arraystretch}{1.5}
  \begin{tabular}{| p{\colAwidth} | p{\colBwidth}|}
    \hline
    \rowcolor[gray]{0.9}
    Number& IM\refstepcounter{instnum}\theinstnum \label{ewat}\label{IM_NeuralStimResponse}\\
    \hline
    Label& \bf Neural Stimulation–Response Instance Model \\
    \hline
    Input& $e(t)$, $\tau_{m,i}$, $v_{m,t}$, \\
    \hline
    Output&$j_{m,t,i}$\\
    \hline
    Description&
    Given preprocessed and time-aligned stimulus features and EEG/MEG recordings (A1), the NCRF pipeline accepts these continuous signals as inputs. \\
    &This instance model specifies the neural stimulation–response relationship at the level of individual cortical dipole sources. \\
    & $j_{m,t,i}$ is the neuro source current at position m and time t. \\
    & $e(t)$ is the stimulus input feature signal. \\
    & $v_{m,t,i}$ is the background source activity.\\
    \hline
    Sources& Citation here \\
    \hline
    Ref.\ By & \iref{epcm}\\
    \hline
  \end{tabular}
\end{minipage}\\

%~\newline

% \subsubsection*{Derivation of ...}

% \plt{The derivation shows how the IM is derived from the TMs/GDs.  In cases
%   where the derivation cannot be described under the Description field, it will
%   be necessary to include this subsection.}

\subsubsection{Input Data Constraints} \label{sec_DataConstraints}

Table~\ref{TblInputVar} shows the data constraints on the input output
variables.  The column for physical constraints gives the physical limitations
on the range of values that can be taken by the variable.  The column for
software constraints restricts the range of inputs to reasonable values.  The
software constraints will be helpful in the design stage for picking suitable
algorithms.  The constraints are conservative, to give the user of the model the
flexibility to experiment with unusual situations.  The column of typical values
is intended to provide a feel for a common scenario.  The uncertainty column
provides an estimate of the confidence with which the physical quantities can be
measured.  This information would be part of the input if one were performing an
uncertainty quantification exercise.

The specification parameters in Table~\ref{TblInputVar} are listed in
Table~\ref{TblSpecParams}.

% \begin{table}[!h]
%   \caption{Input Variables} \label{TblInputVar}
%   \renewcommand{\arraystretch}{1.2}
% \noindent \begin{longtable*}{l l l l c}
%   \toprule
%   \textbf{Var} & \textbf{Physical Constraints} & \textbf{Software Constraints} &
%                              \textbf{Typical Value} & \textbf{Uncertainty}\\
%   \midrule
%   $L$ & $L > 0$ & $L_{\text{min}} \leq L \leq L_{\text{max}}$ & 1.5 \si[per-mode=symbol] {\metre} & 10\%
%   \\
%   \bottomrule
% \end{longtable*}
% \end{table}
\begin{table}[!h]
  \caption{Input Variables} \label{TblInputVar}
  \renewcommand{\arraystretch}{1.3}
  \noindent
  \begin{longtable*}{
      p{0.8cm}
      p{4.5cm}
      p{4.5cm}
      p{3cm}
      p{2cm}
    }
    \toprule
    \textbf{Var}
    & \textbf{Physical Constraints}
    & \textbf{Software Constraints}
    & \textbf{Typical Value}
    & \textbf{Uncertainty} \\
    \midrule

    $L$
    & $L \in \mathbb{Z}^+$
    & $L_{\text{min}} \le L \le L_{\text{max}}$
    & 50 samples
    & None \\

    $e(t)$
    & Finite real-valued signal
    & $|e(t)| \le e_{\text{max}}$
    & Normalized ($[-1,1]$)
    & Feature-dependent \\

    $\tau_{m,i,\ell}$
    & Finite real values
    & $|\tau_{m,i,\ell}| \le \tau_{\text{max}}$
    & Problem-dependent
    & Model-dependent \\

    $v_{m,t,i}$
    & Finite real values
    & Bounded variance
    & Zero-mean
    & High \\

    \bottomrule
  \end{longtable*}
\end{table}

% \noindent
% \begin{description}
% \item[(*)] \plt{you might need to add some notes or clarifications}
% \end{description}

% \begin{table}[!h]
% \caption{Specification Parameter Values} \label{TblSpecParams}
% \renewcommand{\arraystretch}{1.2}
% \noindent \begin{longtable*}{l l}
%   \toprule
%   \textbf{Var} & \textbf{Value} \\
%   \midrule
%   $L_\text{min}$ & 0.1 \si{\metre}\\
%   \bottomrule
% \end{longtable*}
% \end{table}

% \subsubsection{Properties of a Correct Solution} \label{sec_CorrectSolution}

% \noindent
% A correct solution must exhibit \plt{fill in the details}.  \plt{These
%   properties are in addition to the stated requirements.  There is no need to
%   repeat the requirements here.  These additional properties may not exist for
%   every problem.  Examples include conservation laws (like conservation of
%   energy or mass) and known constraints on outputs, which are usually summarized
%   in tabular form.  A sample table is shown in Table~\ref{TblOutputVar}}

% \begin{table}[!h]
% \caption{Output Variables} \label{TblOutputVar}
% \renewcommand{\arraystretch}{1.2}
% \noindent \begin{longtable*}{l l}
%   \toprule
%   \textbf{Var} & \textbf{Physical Constraints} \\
%   \midrule
%   $T_W$ & $T_\text{init} \leq T_W \leq T_C$ (by~\aref{A_charge})
%   \\
%   \bottomrule
% \end{longtable*}
% \end{table}

% \plt{This section is not for test cases or techniques for verification and
%   validation.  Those topics will be addressed in the Verification and Validation
%   plan.}

\newpage
\section{Requirements}

% \plt{The requirements refine the goal statement.  They will make heavy use of
%   references to the instance models.}

This section provides the functional requirements, the business tasks that the
software is expected to complete, and the nonfunctional requirements, the
qualities that the software is expected to exhibit.

\subsection{Functional Requirements}

\noindent
\begin{itemize}

  \item[R\refstepcounter{reqnum}\thereqnum \label{R_Inputs}:]
    The system shall accept a continuous stimulus feature signal $e(t)$ as input to the NCRF pipeline.

  \item[R\refstepcounter{reqnum}\thereqnum \label{R_Calculate}:]
    The system shall compute source-level neural current estimates using the NCRF model (e.g., by applying the learned/estimated response functions to $e(t)$).

  % \item[R\refstepcounter{reqnum}\thereqnum \label{R_VerifyOutput}:]
  %   The system shall provide a mechanism to verify that the computed outputs are consistent with the corresponding inputs and the specified instance models (e.g., by checking dimensions, sampling rates, and expected value ranges).

  \item[R\refstepcounter{reqnum}\thereqnum \label{R_Output}:]
    The system shall output (i) the estimated source currents $j_{m,t}$ and (ii) the NCRF coefficients $\tau_{m,i,\ell}$ in a format suitable for downstream analysis in the existing pipeline.

\end{itemize}

% \plt{Every IM should map to at least one requirement, but not every requirement
%   has to map to a corresponding IM.}

\subsection{Nonfunctional Requirements}

% \plt{List your nonfunctional requirements.  You may consider using a fit
%   criterion to make them verifiable.}
% \plt{The goal is for the nonfunctional requirements to be unambiguous, abstract
%   and verifiable.  This isn't easy to show succinctly, so a good strategy may be
% to give a ``high level'' view of the requirement, but allow for the details to
% be covered in the Verification and Validation document.}
% \plt{An absolute requirement on a quality of the system is rarely needed.  For
%   instance, an accuracy of 0.0101 \% is likely fine, even if the requirement is
%   for 0.01 \% accuracy.  Therefore, the emphasis will often be more on
%   describing now well the quality is achieved, through experimentation, and
%   possibly theory, rather than meeting some bar that was defined a priori.}
% \plt{You do not need an entry for correctness in your NFRs.  The purpose of the
%   SRS is to record the requirements that need to be satisfied for correctness.
%   Any statement of correctness would just be redundant. Rather than discuss
%   correctness, you can characterize how far away from the correct (true)
%   solution you are allowed to be.  This is discussed under accuracy.}

\noindent
\begin{itemize}

\item[NFR1:] \textbf{Accuracy}  
The system shall produce numerically stable and reproducible results that can be verified against established TRF/NCRF benchmarks or reference implementations.

\item[NFR2:] \textbf{Maintainability}  
The system shall be modularly designed such that future extensions and defect fixes require substantially less effort than the initial development.

\item[NFR3:] \textbf{Portability}  
The system shall be implemented as a Python package and be executable on macOS, Windows, and Linux platforms.

\end{itemize}


% \subsection{Rationale}

% \plt{Provide a rationale for the decisions made in the documentation.  Rationale
% should be provided for scope decisions, modelling decisions, assumptions and
% typical values.}

\section{Likely Changes}

\noindent
\begin{itemize}

  \item[LC\refstepcounter{lcnum}\thelcnum\label{LC_OutputFormat}:]
    The output data format may evolve (e.g., adding visualization and summary plots)
    so users can more easily interpret the combined temporal--spatial response data.

  \item[LC\refstepcounter{lcnum}\thelcnum\label{LC_ModelParameters}:]
    Additional model parameters (e.g., new regularization settings or lag-window
    choices) may be introduced as the pipeline is extended.

\end{itemize}

\section{Unlikely Changes}

\noindent
\begin{itemize}

  \item[LC\refstepcounter{lcnum}\thelcnum\label{LC_CoreMethods}:]
    The core analysis approach is unlikely to change; the pipeline will continue to
    support the existing TRF analysis and the integrated NCRF-based response
    functions.

\end{itemize}

\section{Traceability Matrices and Graphs}

The purpose of the traceability matrices is to provide easy references on what
has to be additionally modified if a certain component is changed.  Every time a
component is changed, the items in the column of that component that are marked
with an ``X'' may have to be modified as well.  Table~\ref{Table:trace} shows the
dependencies of theoretical models, general definitions, data definitions, and
instance models with each other. Table~\ref{Table:R_trace} shows the
dependencies of instance models, requirements, and data constraints on each
other. Table~\ref{Table:A_trace} shows the dependencies of theoretical models,
general definitions, data definitions, instance models, and likely changes on
the assumptions.

% \plt{You will have to modify these tables for your problem.}

% \plt{The traceability matrix is not generally symmetric.  If GD1 uses A1, that
%   means that GD1's derivation or presentation requires invocation of A1.  A1
% does not use GD1.  A1 is ``used by'' GD1.}

% \plt{The traceability matrix is challenging to maintain manually.  Please do
% your best.  In the future tools (like Drasil) will make this much easier.}

\afterpage{%
  \begin{landscape}
    \begin{table}[h!]
      \centering
      \begin{tabular}{|c|c|c|c|c|}
        \hline
        & \aref{A1} & \aref{A2} & \aref{A3} & \aref{A4} \\
        \hline
        \tref{TM_NCRF_Model} & & X & X & X \\ \hline
        \dref{GD_StimulusDriven} & & X & X & X \\ \hline
        \dref{GD_BackgroundActivity} & & X & & \\ \hline
        \ddref{DD_NCRF_Coefficients} & & & X & X \\ \hline
        \iref{IM_NeuralStimResponse} & & X & X & X \\ \hline
        \lcref{LC_OutputFormat} & & & & \\ \hline
        \lcref{LC_ModelParameters} & & & X & X \\ \hline
        \lcref{LC_CoreMethods} & & & & \\ \hline
      \end{tabular}
      \caption{Traceability Matrix Showing the Connections Between Assumptions and Other Items}
      \label{Table:A_trace}
    \end{table}
  \end{landscape}
}

\begin{table}[h!]
  \centering
  \begin{tabular}{|c|c|c|c|c|}
    \hline
    & \tref{TM_NCRF_Model} & \dref{GD_StimulusDriven} & \dref{GD_BackgroundActivity} & \ddref{DD_NCRF_Coefficients} \\
    \hline
    \tref{TM_NCRF_Model} & & & & \\ \hline
    \dref{GD_StimulusDriven} & X & & & \\ \hline
    \dref{GD_BackgroundActivity} & X & & & \\ \hline
    \ddref{DD_NCRF_Coefficients} & X & & & \\ \hline
    \iref{IM_NeuralStimResponse} & X & X & X & X \\ \hline
  \end{tabular}
  \caption{Traceability Matrix Showing the Connections Between Items of Different Sections}
  \label{Table:trace}
\end{table}

\begin{table}[h!]
  \centering
  \begin{tabular}{|c|c|c|c|c|}
    \hline
    & \rref{R_Inputs} & \rref{R_Calculate} & \rref{R_Output} & \iref{IM_NeuralStimResponse} \\
    \hline
    \rref{R_Inputs} & & & & X \\ \hline
    \rref{R_Calculate} & & & & X \\ \hline
    \rref{R_Output} & & & & X \\ \hline
    \iref{IM_NeuralStimResponse} & X & X & X & \\ \hline
  \end{tabular}
  \caption{Traceability Matrix Showing the Connections Between Requirements and Instance Models}
  \label{Table:R_trace}
\end{table}

The purpose of the traceability graphs is also to provide easy references on
what has to be additionally modified if a certain component is changed.  The
arrows in the graphs represent dependencies. The component at the tail of an
arrow is depended on by the component at the head of that arrow. Therefore, if a
component is changed, the components that it points to should also be changed.

% \begin{figure}[h!]
%   \begin{center}
%     %\rotatebox{-90}
%     {
%       \includegraphics[width=\textwidth]{ATrace.png}
%     }
%     \caption{\label{Fig_ATrace} Traceability Matrix Showing the Connections Between Items of Different Sections}
%   \end{center}
% \end{figure}

% \begin{figure}[h!]
%   \begin{center}
%     %\rotatebox{-90}
%     {
%       \includegraphics[width=0.7\textwidth]{RTrace.png}
%     }
%     \caption{\label{Fig_RTrace} Traceability Matrix Showing the Connections Between Requirements, Instance Models, and Data Constraints}
%   \end{center}
% \end{figure}

% \section{Development Plan}

% \plt{This section is optional.  It is used to explain the plan for developing
%   the software.  In particular, this section gives a list of the order in which
%   the requirements will be implemented.  In the context of a course  this is
%   where you can indicate which requirements will be implemented as part of the
%   course, and which will be ``faked'' as future work.  This section can be
%   organized as a prioritized list of requirements, or it could should the
% requirements that will be implemented for ``phase 1'', ``phase 2'', etc.}

\section{Values of Auxiliary Constants}

\begin{itemize}
  \item \textbf{Sampling rate}: EEG/MEG and stimulus signals are assumed to be sampled at a fixed rate (e.g., 1000 Hz).
  \item \textbf{Lag window}: The temporal response functions are defined over a fixed lag window (e.g., 0–500 ms).
  \item \textbf{Lag resolution}: Lags are discretized according to the sampling interval.
\end{itemize}

% \plt{Show the values of the symbolic parameters introduced in the report.}

% \plt{The definition of the requirements will likely call for SYMBOLIC\_CONSTANTS.
% Their values are defined in this section for easy maintenance.}

% \plt{The value of FRACTION, for the Maintainability NFR would be given here.}

\newpage

\bibliographystyle {plainnat}
\bibliography {../../refs/References}
% \newpage

% \noindent \plt{The following is not part of the template, just some things to consider
% when filing in the template.}

% \noindent \plt{Grammar, flow and \LaTeX advice:
%   \begin{itemize}
%     \item For Mac users \texttt{*.DS\_Store} should be in \texttt{.gitignore}
%     \item \LaTeX{} and formatting rules
%       \begin{itemize}
%         \item Variables are italic, everything else not, includes subscripts (link to
%           document)
%           \begin{itemize}
%             \item \href{https://physics.nist.gov/cuu/pdf/typefaces.pdf}{Conventions}
%             \item Watch out for implied multiplication
%           \end{itemize}
%         \item Use BibTeX
%         \item Use cross-referencing
%       \end{itemize}
%     \item Grammar and writing rules
%       \begin{itemize}
%         \item Acronyms expanded on first usage (not just in table of acronyms)
%         \item ``In order to'' should be ``to''
%       \end{itemize}
%   \end{itemize}}

% \noindent \plt{Advice on using the template:
%   \begin{itemize}
%     \item Difference between physical and software constraints
%     \item Properties of a correct solution means \emph{additional} properties, not
%       a restating of the requirements (may be ``not applicable'' for your problem).
%       If you have a table of output constraints, then these are properties of a
%       correct solution.
%     \item Assumptions have to be invoked somewhere
%     \item ``Referenced by'' implies that there is an explicit reference
%     \item Think of traceability matrix, list of assumption invocations and list of
%       reference by fields as automatically generatable
%     \item If you say the format of the output (plot, table etc), then your
%       requirement could be more abstract
%   \end{itemize}
% }

% \newpage{}
% \section*{Appendix --- Reflection}

% \wss{Not required for CAS 741}

% The information in this section will be used to evaluate the team members on the
% graduate attribute of Lifelong Learning.

% \input{../Reflection.text}

% \input{../SRS_Reflection.text}

\end{document}
