\documentclass[12pt, titlepage]{article}

\usepackage{booktabs}
\usepackage{tabularx}
\usepackage{hyperref}
\hypersetup{
    colorlinks,
    citecolor=blue,
    filecolor=black,
    linkcolor=red,
    urlcolor=blue
}
\usepackage[round]{natbib}

\input{../Comments.text}
\input{../Common.text}

\renewcommand{\progname}{Integrating NCRF into existing pipeline}

\begin{document}

\title{System Verification and Validation Plan for \progname{}} 
\author{Yanyu Wu}
\date{\today}
	
\maketitle

\pagenumbering{roman}

\section*{Revision History}

\begin{tabularx}{\textwidth}{p{3cm}p{2cm}X}
\toprule {\bf Date} & {\bf Version} & {\bf Notes}\\
\midrule
Feb 4 & 1.0 & Initial draft\\
\bottomrule
\end{tabularx}

% ~\\
% \wss{The intention of the VnV plan is to increase confidence in the software.
% However, this does not mean listing every verification and validation technique
% that has ever been devised.  The VnV plan should also be a \textbf{feasible}
% plan. Execution of the plan should be possible with the time and team available.
% If the full plan cannot be completed during the time available, it can either be
% modified to ``fake it'', or a better solution is to add a section describing
% what work has been completed and what work is still planned for the future.}

% \wss{The VnV plan is typically started after the requirements stage, but before
% the design stage.  This means that the sections related to unit testing cannot
% initially be completed.  The sections will be filled in after the design stage
% is complete.  the final version of the VnV plan should have all sections filled
% in.}

\newpage

\tableofcontents

\listoftables
\wss{Remove this section if it isn't needed}

\listoffigures
\wss{Remove this section if it isn't needed}

\newpage

\section{Symbols, Abbreviations, and Acronyms}

% \renewcommand{\arraystretch}{1.2}
% \begin{tabular}{l l} 
%   \toprule		
%   \textbf{symbol} & \textbf{description}\\
%   \midrule 
%   T & Test\\
%   \bottomrule
% \end{tabular}\\

% \wss{symbols, abbreviations, or acronyms --- you can simply reference the SRS
%   \citep{SRS} tables, if appropriate}

% \wss{Remove this section if it isn't needed}

The definition of symbols, abbreviations and acronyms is as same as those in my SRS documents.

\newpage

\pagenumbering{arabic}

This document provides an introductory blurb and roadmap of the NCRF pipeline Verification and Validation plan.

\section{General Information}

\subsection{Summary}

% \wss{Say what software is being tested.  Give its name and a brief overview of
%   its general functions.}
Integrating NCRF into existing pipeline is intended to solve the error propagation issue of
traditional TRF-based pipeline. Using NCRF could allow spatial and temporal response
properties to be estimated jointly, instead of sequentially.
\subsection{Objectives}
\textbf{Primary Objectives}
\begin{itemize}
  \item Build confidence that the NCRF integration satisfies the SRS functional requirements.
  \item Verify numerical correctness on small, controlled datasets by matching expected outputs within defined tolerances.
  \item Demonstrate reproducibility of results across repeated runs with fixed seeds and documented configurations.
  \item Guarantee key non-functional requirements (e.g., reliability, traceability).
\end{itemize}
\textbf{Out of Scope}
\begin{itemize}
  \item Internal correctness or performance of third-party libraries (e.g., Eelbrain, MNE); assume they are already verified by their maintainers.
  \item Large-scale performance and scalability benchmarking on very large datasets.
\end{itemize}
% \wss{State what is intended to be accomplished.  The objective will be around
%   the qualities that are most important for your project.  You might have
%   something like: ``build confidence in the software correctness,''
%   ``demonstrate adequate usability.'' etc.  You won't list all of the qualities,
%   just those that are most important.}

% \wss{You should also list the objectives that are out of scope.  You don't have 
% the resources to do everything, so what will you be leaving out.  For instance, 
% if you are not going to verify the quality of usability, state this.  It is also 
% worthwhile to justify why the objectives are left out.}

% \wss{The objectives are important because they highlight that you are aware of 
% limitations in your resources for verification and validation.  You can't do everything, 
% so what are you going to prioritize?  As an example, if your system depends on an 
% external library, you can explicitly state that you will assume that external library 
% has already been verified by its implementation team.}

\subsection{Extras}

% \wss{Summarize the extras (if any) that were tackled by this project.  Extras
% can include usability testing, code walkthroughs, user documentation, formal
% proof, GenderMag personas, Design Thinking, etc.  Extras should have already
% been approved by the course instructor as included in your problem statement.
% You can use a pull request to update your extras (in TeamComposition.csv or
% Repos.csv) if your plan changes as a result of the VnV planning exercise.}
Two project extras are planned and were declared at the start of the course:
\textbf{User Manual} and \textbf{Code Walkthrough}. 

The User Manual will provide task-oriented guidance (installation, input preparation, running the pipeline,
and interpreting outputs) and will be validated through checklist-based review
for clarity, completeness, and consistency with the SRS and system behavior. The

Code Walkthrough will be a structured inspection session using a checklist to
verify coding standards, traceability from requirements to implementation, and
consistency with the architecture and detailed design documents. Findings from
both activities will be recorded as actionable issues and tracked to resolution.

\subsection{Relevant Documentation}

% \wss{Reference relevant documentation.  This will definitely include your SRS
%   and your other project documents (design documents, like MG, MIS, etc).  You
%   can include these even before they are written, since by the time the project
%   is done, they will be written.  You can create BibTeX entries for your
%   documents and within those entries include a hyperlink to the documents.}

The Software Requirements Specification (SRS) is the primary source for
verification targets; all test cases and traceability links originate from it
(\href{../SRS/SRS.tex}{SRS}).

The Module Guide (MG) defines the system
decomposition and responsibilities used to plan design-level reviews and
coverage (\href{../Design/SoftArchitecture/MG.tex}{MG}).

The Module Interface Specification (MIS) provides detailed module interfaces and expected behaviors
to support unit-level verification (\href{../Design/SoftDetailedDes/MIS.tex}{MIS}).

% \citet{SRS}

% \wss{Don't just list the other documents.  You should explain why they are relevant and 
% how they relate to your VnV efforts.}

\section{Plan}
This section provides a roadmap of the verification and validation activities planned for the project.
% \wss{Introduce this section.  You can provide a roadmap of the sections to
%   come.}

\subsection{Verification and Validation Team}

\begin{table}[h!]
\centering
\begin{tabularx}{\textwidth}{p{0.18\textwidth} p{0.22\textwidth} X}
\toprule
\textbf{Member} & \textbf{Role} & \textbf{Responsibilities} \\
\midrule
Yanyu Wu & Developer / V\&V Lead & Owns the V\&V plan, executes verification activities, and documents results and issues. \\
Dr.Brodbeck & Reviewer & Provides high-level guidance on direction and priorities. \\
Dr.Smith & Course Instructor & Provides feedback on the V\&V plan and documentation. \\
Zhangwenchi Li & Domain Expert & Conducts code review and provides domain feedback on code and documentation. \\
\bottomrule
\end{tabularx}
\caption{Verification and Validation Team}
\label{tab:vv_team}
\end{table}

% \wss{Your teammates.  Maybe your supervisor.
%   You should do more than list names.  You should say what each person's role is
%   for the project's verification.  A table is a good way to summarize this information.}

\subsection{SRS Verification}

% \wss{List any approaches you intend to use for SRS verification.  This may
%   include ad hoc feedback from reviewers, like your classmates (like your
%   primary reviewer), or you may plan for something more rigorous/systematic.}
%
% \wss{If you have a supervisor for the project, you shouldn't just say they will
% read over the SRS.  You should explain your structured approach to the review.
% Will you have a meeting?  What will you present?  What questions will you ask?
% Will you give them instructions for a task-based inspection?  Will you use your
% issue tracker?}
%
% \wss{Maybe create an SRS checklist?}
SRS verification will be done through a structured review and a requirements
checklist:
\begin{itemize}
  \item Domain expert review for scientific validity and consistency with NCRF pipeline goals.
  \item Supervisor review for clarity, completeness, and alignment with project scope.
  \item Developer self-check to ensure functional and non-functional requirements are specific, testable, and traceable.
\end{itemize}

\subsection{Design Verification}

% \wss{Plans for design verification}

% \wss{The review will include reviews by your classmates}
%
% \wss{Create a checklists?}
Design verification will combine structured peer review and checklist-based inspection:
\begin{itemize}
  \item Classmate / peer design review: scheduled walkthrough sessions where classmates provide feedback on the design documents (MG and MIS).
  \item Design checklists: ensure each SRS requirement is addressed at the design level, with checks for modularity, clarity, and traceability to requirements.
\end{itemize}

\subsection{Verification and Validation Plan Verification}

% \wss{The verification and validation plan is an artifact that should also be
% verified.  Techniques for this include review and mutation testing.}
%
% \wss{The review will include reviews by your classmates}
%
% \wss{Create a checklists?}
The V\&V plan will be verified to ensure it is complete, feasible, and aligned
with the SRS:
\begin{itemize}
  \item Peer review to check clarity, coverage, and feasibility of the planned activities.
  \item V\&V checklist inspection to ensure required sections and details are present.
  \item Traceability spot-check to confirm each SRS requirement has at least one planned verification method.
\end{itemize}

\subsection{Implementation Verification}

% \wss{You should at least point to the tests listed in this document and the unit
%   testing plan.}
%
% \wss{In this section you would also give any details of any plans for static
%   verification of the implementation.  Potential techniques include code
%   walkthroughs, code inspection, static analyzers, etc.}
%
% \wss{The final class presentation in CAS 741 could be used as a code
% walkthrough.  There is also a possibility of using the final presentation (in
% CAS741) for a partial usability survey.}
Implementation verification will focus on test execution, requirements coverage,
and inspection:
\begin{itemize}
  \item Execute planned test cases (unit and system-level) and record results against expected outputs.
  \item Verify coverage by checking each SRS requirement is implemented and has at least one corresponding test.
  \item Conduct code walkthroughs to inspect critical logic, edge cases, and consistency with the design (MG/MIS).
\end{itemize}

\subsection{Automated Testing and Verification Tools}

% \wss{What tools are you using for automated testing.  Likely a unit testing
%   framework and maybe a profiling tool, like ValGrind.  Other possible tools
%   include a static analyzer, make, continuous integration tools, test coverage
%   tools, etc.  Explain your plans for summarizing code coverage metrics.
%   Linters are another important class of tools.  For the programming language
%   you select, you should look at the available linters.  There may also be tools
%   that verify that coding standards have been respected, like flake9 for
%   Python.}

% \wss{If you have already done this in the development plan, you can point to
% that document.}

% \wss{The details of this section will likely evolve as you get closer to the
%   implementation.}
Automated verification will use a small set of standard Python tools and CI:
\begin{itemize}
  \item Continuous Integration (CI) to run tests and checks on each update to the repository.
  \item Unit testing with `pytest` for functional correctness of modules and utilities.
  \item Coverage tracking with `Coverage.py` to summarize test coverage.
\end{itemize}

\subsection{Software Validation}
Software validation will focus on confirming that the pipeline meets the intended scientific use and produces meaningful outputs:
\begin{itemize}
  \item Supervisor review of results and workflow to confirm the system addresses the project goals.
  \item Domain expert feedback on the interpretability and scientific plausibility of NCRF outputs.
  \item Validation on a small, real dataset to ensure outputs match expected patterns and are usable for downstream analysis.
\end{itemize}

% \wss{If there is any external data that can be used for validation, you should
%   point to it here.  If there are no plans for validation, you should state that
%   here.}

% \wss{You might want to use review sessions with the stakeholder to check that
% the requirements document captures the right requirements.  Maybe task based
% inspection?}

% \wss{For those capstone teams with an external supervisor, the Rev 0 demo should 
% be used as an opportunity to validate the requirements.  You should plan on 
% demonstrating your project to your supervisor shortly after the scheduled Rev 0 demo.  
% The feedback from your supervisor will be very useful for improving your project.}

% \wss{For teams without an external supervisor, user testing can serve the same purpose 
% as a Rev 0 demo for the supervisor.}

% \wss{This section might reference back to the SRS verification section.}

\section{System Tests}
This section describes the system tests planned for the project.
% \wss{There should be text between all headings, even if it is just a roadmap of
% the contents of the subsections.}

\subsection{Tests for Functional Requirements}
This sections defines the test cases for key funtional areas of Integrating NCRF into existing pipeline. The goal for the test cases are checking the project correctly calculated the result using NCRF and the functions are be integrated into the existing pipeline correctly.

The test cases are derived from the functional requirements in the Software Requirements Specification (SRS).

% \wss{Subsets of the tests may be in related, so this section is divided into
%   different areas.  If there are no identifiable subsets for the tests, this
%   level of document structure can be removed.}

% \wss{Include a blurb here to explain why the subsections below
%   cover the requirements.  References to the SRS would be good here.}
The subsections below map directly to the SRS functional requirements:
input handling (R\ref{R-Inputs}), NCRF computation (R\ref{R-Calculate}), and
output generation for downstream analysis (R\ref{R-Output}). Each test area
is designed to verify that the implemented pipeline satisfies these specific
requirements.

% \wss{It would be nice to have a blurb here to explain why the subsections below
%   cover the requirements.  References to the SRS would be good here.  If a section
%   covers tests for input constraints, you should reference the data constraints
%   table in the SRS.}

\subsubsection{Input Handling and API Selection (R\ref{R-Inputs})}
This area verifies that the pipeline accepts required inputs and exposes an
API that selects NCRF estimation as described in issue \#50.
		
\paragraph{Input and Estimator Selection}

\begin{enumerate}

\item{FR-INPUT-01\\}

Control: Automatic
					
Initial State: BIDS dataset and stimulus features are available; configuration is valid.
					
Input: Valid stimulus feature signal $e(t)$ and EEG/MEG data; call \texttt{load\_trf(estimator="ncrf")}.
					
Output: The NCRF pipeline is invoked without input/parameter errors; data and parameters are accepted.

Test Case Derivation: R\ref{R-Inputs} requires the system to accept $e(t)$ as input.
					
How test will be performed: Run the pipeline with a small BIDS dataset and log successful NCRF initialization.
					
\item{FR-API-02\\}

Control: Automatic
					
Initial State: Same as FR-INPUT-01.
					
Input: Call \texttt{load\_trf(estimator="unknown")}.
					
Output: A clear error is raised indicating invalid estimator and listing supported options.

Test Case Derivation: The API must validate estimator selection to prevent invalid execution paths.

How test will be performed: Invoke the call and verify the error message and exit behavior.

\item{FR-API-03\\}

Control: Automatic
					
Initial State: Existing TRF workflow configuration.
					
Input: Call \texttt{load\_trf()} without specifying estimator.
					
Output: Default (boosting) behavior is preserved; no regression in baseline workflow.

Test Case Derivation: Issue \#50 emphasizes API unification without breaking existing TRF usage.

How test will be performed: Run a known TRF configuration and compare outputs to baseline.

\end{enumerate}

\subsubsection{NCRF Computation (R\ref{R-Calculate})}
This area verifies that NCRF estimation executes and produces valid outputs.

\paragraph{NCRF Execution Sanity}

\begin{enumerate}

\item{FR-CALC-01\\}

Control: Automatic

Initial State: Small, controlled dataset with fixed configuration.

Input: Valid $e(t)$ and data; NCRF estimator selected.

Output: Estimated $j_{m,t}$ and $\tau_{m,i,\ell}$ are produced with expected dimensions.

Test Case Derivation: R\ref{R-Calculate} requires computation of source-level neural current estimates.

How test will be performed: Run NCRF on a small dataset and verify output shapes against model settings.

\item{FR-CALC-02\\}

Control: Automatic

Initial State: Same as FR-CALC-01 with fixed random seed.

Input: Repeat the same NCRF run twice.

Output: Results are numerically consistent across runs (within tolerance).

Test Case Derivation: Supports reproducibility expectations aligned with NFR1 while validating computation.

How test will be performed: Compare outputs from repeated runs using relative error or norm.

\end{enumerate}

\subsubsection{Output Generation and Downstream Use (R\ref{R-Output})}
This area verifies output content and compatibility with downstream analysis.

\paragraph{Output Format and Integration}

\begin{enumerate}

\item{FR-OUT-01\\}

Control: Automatic

Initial State: Successful NCRF computation completed.

Input: NCRF output objects/files.

Output: Outputs include $j_{m,t}$ and $\tau_{m,i,\ell}$ in the expected structure and format.

Test Case Derivation: R\ref{R-Output} requires both outputs in a format suitable for downstream analysis.

How test will be performed: Inspect output fields and validate metadata/shape consistency.

\item{FR-OUT-02\\}

Control: Automatic

Initial State: Outputs from FR-OUT-01 available.

Input: Load NCRF outputs into the existing downstream pipeline step.

Output: Downstream analysis successfully loads and processes the outputs without format errors.

Test Case Derivation: R\ref{R-Output} requires interoperability with the existing pipeline.

How test will be performed: Execute downstream loading script and verify completion.

\end{enumerate}

...

\subsection{Tests for Nonfunctional Requirements}

\wss{The nonfunctional requirements for accuracy will likely just reference the
  appropriate functional tests from above.  The test cases should mention
  reporting the relative error for these tests.  Not all projects will
  necessarily have nonfunctional requirements related to accuracy.}

\wss{For some nonfunctional tests, you won't be setting a target threshold for
passing the test, but rather describing the experiment you will do to measure
the quality for different inputs.  For instance, you could measure speed versus
the problem size.  The output of the test isn't pass/fail, but rather a summary
table or graph.}

\wss{Tests related to usability could include conducting a usability test and
  survey.  The survey will be in the Appendix.}

\wss{Static tests, review, inspections, and walkthroughs, will not follow the
format for the tests given below.}

\wss{If you introduce static tests in your plan, you need to provide details.
How will they be done?  In cases like code (or document) walkthroughs, who will
be involved? Be specific.}

\subsubsection{Accuracy and Reproducibility (NFR1)}
		
\paragraph{Numerical Accuracy Against Reference}

\begin{enumerate}

\item{NFR-ACC-01\\}

Type: Dynamic, Automatic
					
Initial State: Reference TRF/NCRF benchmark or baseline outputs are available.
					
Input/Condition: Run NCRF on the reference dataset with documented parameters.
					
Output/Result: Output matches the reference within an agreed tolerance (relative error or norm).
					
How test will be performed: Compare $j_{m,t}$ and $\tau_{m,i,\ell}$ to reference results and report error metrics.
					
\item{NFR-ACC-02\\}

Type: Dynamic, Automatic
					
Initial State: Fixed random seed and configuration.
					
Input/Condition: Repeat NCRF estimation twice.
					
Output/Result: Outputs are numerically consistent across runs within tolerance.
					
How test will be performed: Compute relative error or norm between repeated runs and report results.

\end{enumerate}

\subsubsection{Maintainability (NFR2)}
		
\paragraph{Modularity and Code Structure Review}

\begin{enumerate}

\item{NFR-MAINT-01\\}

Type: Static, Manual
					
Initial State: MG/MIS and codebase are available for review.
					
Input/Condition: Inspect module structure and interfaces for separation of concerns.
					
Output/Result: NCRF integration is localized to well-defined modules, with minimal changes to unrelated code.
					
How test will be performed: Checklist-based review of module boundaries and interface usage (MG/MIS checklist).

\end{enumerate}

\subsubsection{Portability (NFR3)}
		
\paragraph{Cross-Platform Execution}

\begin{enumerate}

\item{NFR-PORT-01\\}

Type: Dynamic, Automatic
					
Initial State: Supported environments available (macOS, Windows, Linux) or CI matrix configured.
					
Input/Condition: Run a minimal NCRF workflow on each platform.
					
Output/Result: The pipeline executes without platform-specific errors.
					
How test will be performed: Execute automated tests in CI (or local runs) and record pass/fail on each OS.

\end{enumerate}

\subsection{Traceability Between Test Cases and Requirements}

\wss{Provide a table that shows which test cases are supporting which
  requirements.}

\section{Unit Test Description}

\wss{This section should not be filled in until after the MIS (detailed design
  document) has been completed.}

\wss{Reference your MIS (detailed design document) and explain your overall
philosophy for test case selection.}  

\wss{To save space and time, it may be an option to provide less detail in this section.  
For the unit tests you can potentially layout your testing strategy here.  That is, you 
can explain how tests will be selected for each module.  For instance, your test building 
approach could be test cases for each access program, including one test for normal behaviour 
and as many tests as needed for edge cases.  Rather than create the details of the input 
and output here, you could point to the unit testing code.  For this to work, you code 
needs to be well-documented, with meaningful names for all of the tests.}

\subsection{Unit Testing Scope}

\wss{What modules are outside of the scope.  If there are modules that are
  developed by someone else, then you would say here if you aren't planning on
  verifying them.  There may also be modules that are part of your software, but
  have a lower priority for verification than others.  If this is the case,
  explain your rationale for the ranking of module importance.}

\subsection{Tests for Functional Requirements}

\wss{Most of the verification will be through automated unit testing.  If
  appropriate specific modules can be verified by a non-testing based
  technique.  That can also be documented in this section.}

\subsubsection{Module 1}

\wss{Include a blurb here to explain why the subsections below cover the module.
  References to the MIS would be good.  You will want tests from a black box
  perspective and from a white box perspective.  Explain to the reader how the
  tests were selected.}

\begin{enumerate}

\item{test-id1\\}

Type: \wss{Functional, Dynamic, Manual, Automatic, Static etc. Most will
  be automatic}
					
Initial State: 
					
Input: 
					
Output: \wss{The expected result for the given inputs}

Test Case Derivation: \wss{Justify the expected value given in the Output field}

How test will be performed: 
					
\item{test-id2\\}

Type: \wss{Functional, Dynamic, Manual, Automatic, Static etc. Most will
  be automatic}
					
Initial State: 
					
Input: 
					
Output: \wss{The expected result for the given inputs}

Test Case Derivation: \wss{Justify the expected value given in the Output field}

How test will be performed: 

\item{...\\}
    
\end{enumerate}

\subsubsection{Module 2}

...

\subsection{Tests for Nonfunctional Requirements}

\wss{If there is a module that needs to be independently assessed for
  performance, those test cases can go here.  In some projects, planning for
  nonfunctional tests of units will not be that relevant.}

\wss{These tests may involve collecting performance data from previously
  mentioned functional tests.}

\subsubsection{Module ?}
		
\begin{enumerate}

\item{test-id1\\}

Type: \wss{Functional, Dynamic, Manual, Automatic, Static etc. Most will
  be automatic}
					
Initial State: 
					
Input/Condition: 
					
Output/Result: 
					
How test will be performed: 
					
\item{test-id2\\}

Type: Functional, Dynamic, Manual, Static etc.
					
Initial State: 
					
Input: 
					
Output: 
					
How test will be performed: 

\end{enumerate}

\subsubsection{Module ?}

...

\subsection{Traceability Between Test Cases and Modules}

\wss{Provide evidence that all modules have been considered.}
				
\bibliographystyle{plainnat}

\bibliography{../../refs/References}

\newpage

\section{Appendix}

This is where you can place additional information.

\subsection{Symbolic Parameters}

The definition of the test cases will call for SYMBOLIC\_CONSTANTS.
Their values are defined in this section for easy maintenance.

\subsection{Usability Survey Questions?}

\wss{This is a section that would be appropriate for some projects.}

\newpage{}
\section*{Appendix --- Reflection}

\wss{This section is not required for CAS 741}

The information in this section will be used to evaluate the team members on the
graduate attribute of Lifelong Learning.

\input{../Reflection.text}

\begin{enumerate}
  \item What went well while writing this deliverable? 
  \item What pain points did you experience during this deliverable, and how
    did you resolve them?
  \item What knowledge and skills will the team collectively need to acquire to
  successfully complete the verification and validation of your project?
  Examples of possible knowledge and skills include dynamic testing knowledge,
  static testing knowledge, specific tool usage, Valgrind etc.  You should look to
  identify at least one item for each team member.
  \item For each of the knowledge areas and skills identified in the previous
  question, what are at least two approaches to acquiring the knowledge or
  mastering the skill?  Of the identified approaches, which will each team
  member pursue, and why did they make this choice?
\end{enumerate}

\end{document}
