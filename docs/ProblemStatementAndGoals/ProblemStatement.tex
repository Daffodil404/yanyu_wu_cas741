\documentclass{article}
\newcommand{\authorname}{Yanyu Wu}

\usepackage{tabularx}
\usepackage{booktabs}

\title{Problem Statement and Goals\\Integrating NCRFs Into Pipeline}

\author{\authorname}

\date{January 23 2026}


\begin{document}

\maketitle

\begin{table}[hp]
\caption{Revision History} \label{TblRevisionHistory}
\begin{tabularx}{\textwidth}{llX}
\toprule
\textbf{Date} & \textbf{Developer(s)} & \textbf{Change}\\
\midrule
January 15 2026 & Yanyu Wu & Initial Draft\\
January 18 2026 & Yanyu Wu & Refined based on feedback \\
January 23 2026 & Yanyu Wu & Concept and goal adjust \\
\bottomrule
\end{tabularx}
\end{table}
\section{Problem Statement}

EEG (Electroencephalography) and MEG(Magnetoencephalography) data are widely used in neuroscience area, to study the brain activity with high temporal resolution. In experiments involving continuous stimulimsuch as natural speech or music, researchers are interested in understanding both when neural responses occur and where in the brain these responses are generated. 

Temporal Response Functions (TRFs) are encoding model, describing how the continous stimuli will be mapped to neural signal by brain. Current TRF-based analysis pipelines follow a two-stage approach: (1) estimating neural source activity from sensor of EEG or MEG. (2)estimating TRFs on these fixed source estimates. A key limitation for this method is that uncertainty and errors in spatial estimation can directly affect the estimated TRFs.

Neuro-Current Response Functions (NCRFs) are proposed to address this limitation by providing a unified approach that jointly estimates spatial and temporal aspects of neural responses. Instead of separating source localization and temporal modeling, NCRFs treat each cortical source as a linear filter that maps continuous stimuli directly to neural current activity.

\subsection{Problem}

The main problem addressed in this project is that existing TRF method based pipeline separate spatial and temporal modeling, which may cause error propagation.

This project aims to integrating NCRFs into existing purely TRF-based pipeline, allowing spatial and temporal response properties to be estimated jointly rather than sequentially. 

\subsection{Inputs and Outputs}
\textbf{Inputs}
\begin{itemize}
    \item Combination of different time series stimuli, such as acoustic envelope, word frequency etc.
    \item Data from Brain Imaging Dataset (BIDS).
    \item Configuration parameters, such as file paths, stimulus duration, model settings.
\end{itemize}
\textbf{Outputs}
\begin{itemize}
    \item Spitial information, indicating where in the brain stimulus-related activity occurs.
    \item Temporal information, describing how neural responses evolves over time.
    \item Result visualization to indicate temporal and spatial response.
\end{itemize}

\subsection{Stakeholders}
Stakeholders include lab members, researchers, and students who use the pipeline to conduct EEG/MEG data analysis and model evaluation.

\subsection{Environment}
\begin{itemize}
\item The pipeline can be executed on modern operating systems, including macOS, Windows, and Linux. 
\item The project will be developed based on Python programming language.
\item The environment includes the Eelbrain library, which supports EEG/MEG data analysis, modeling and result visualization.
\end{itemize}

\section{Goals}
\begin{itemize}
    \item From sensor or channel TRF to cortical TRF, getting the response map of whole cortex response. 
    \item Reduce the inaccuracy from two-step methods.
    \item Improve the explainability in neuro-science.
\end{itemize}

\section{Stretch Goals}
\begin{itemize}
    \item Provide an interactive user interface GUI for this pipeline.
    \item Enable direct comparison between traditional TRF outputs and NCRF-based outputs within the same pipeline.
\end{itemize}

\section{Extras}

Although this project is not a research project, it still requires a certain level of knowledge in neuroscience and related data processing methods; otherwise, users might be confused about the pipeline's functionality.

\begin{itemize}

\item \textbf{User documentation}: Provide detailed instructions, including installation, NCRF analysis, execution methods.
\item \textbf{Code Walkthrough}: Provide a structured walkthrough of the pipeline implementation, explaining the overall architecture, key modules, data flow, and how NCRF components are integrated into the existing TRF-based pipeline.
\end{itemize}

\end{document}